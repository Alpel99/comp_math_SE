\section{The Stokes Equations in NGSolve}

The Stokes Equation, a linear partial differential equation, can describe a stationary incompressible Newtonian fluid
with high viscosities and low Reynolds numbers. It is defined below in equation \ref{stokes_PDE}:
\null

\begin{equation}\label{stokes_PDE}
    \begin{aligned}
    -\mu \Delta u + \nabla p &= f \\
    \mathrm{div} \, u &= 0 \\
    \end{aligned}
\end{equation}

\null

Where $\mu \in \mathbb{R}$ is the viscosity constant and $f$ the data (e.g. source). 
The problem yields the vectorial velocity field $u:\Omega \rightarrow \mathbb{R}^d$ and 
the scalar pressure field $p:\Omega \rightarrow \mathbb{R}$. In order to solve the Stokes equation with the Finite Element Method in NGSolve,
it needs to be transformed to the weak formulation, where the solutions $u$ and $p$ are linear combinations of basis functions in a Sobolev space.
See Faustmann\cite{lecture_notes_faustmann_numPDE} Chapter 3 for further elaborations on Sobolev spaces. The weak formulation can be derrived by
multipling the now called trial-functions $u$ and $p$ with test-functions $v$ and $q$, perform transformations and integrate them. The test-functions have to fulfil certain
conditions to permit the transformations in order to arrive at a weak problem with 
linear convergence rates, see Faustmann\cite{lecture_notes_faustmann_numPDE}: \\

Find $u \in [H^1_0(\Omega)]^d$ and $p \in L^2(\Omega)$ such that
\begin{equation}\label{weak_stokes_PDE}
    \begin{aligned}
    &\int_{\Omega} \nabla u : \nabla v \, dx + \int_{\Omega} \mathrm{div}(v)p \, dx &=& \int_{\Omega}f v \, dx &\forall& v \in [H^1_0(\Omega)]^d \\
    &\int_{\Omega} \mathrm{div}(u)q \, dx &=& \, 0   &\forall& q \in L^2(\Omega)
    \end{aligned}
\end{equation}

The

