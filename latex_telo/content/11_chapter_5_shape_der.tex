
\section{Shape Derivative in NGSolve}

\subsection{Derrivative of the Augmented Lagrangian}

After the Augmented Lagrangian is set up accordingly, the derrivative has to bo calculated. 

This derrivative is then used in our code for \ref{basic_lagrangian} this is done by using ngsolves implementation of \fun{DiffShape()}. 
This way we only have to specify what is written in \ref{basic_lagrangian} and the rest is done by ngsolve on its own.

This does not work for the side constraints, because squaring and subtracting with these defined integrals is not straight forward.
To calculate these by hand, one arrives at the following formulas for the volume:

\begin{equation}\label{eq:constraints_vol}
	%L(\Omega,u,p) += 
	2\alpha (vol-vol_0)*div(X)
\end{equation}

and the barycenter in direction x:

\begin{equation}\label{eq:constraints_bc}
	%L(\Omega,u,p) += 
	2\beta (bc_x-bc_{x0})\cdot\left(-\frac{1}{vol^2}\int div(X)\mathrm{dx} \cdot \int x \mathrm{dx} + \frac{1}{vol} \cdot (\int div(X)*x \mathrm{dx} + \int X_i\mathrm{dx})\right)
\end{equation}

where x is the current direction variable and $X_i$ is the deformation field in the i'th (current) direction.