\section{Gradient Descent Method Iterations}

Since the minimization is done with the Gradient Descent Method iteratively, one has to keep track of the previous 
deformations. This is done in NGSolve by adding the parts of \var{gfX} to another 
variable called \var{gfset}. This \var{gfset} is then always used to call \fun{SetDeformation()}
on the mesh. With each call, this adds the \var{gfset} onto the mesh. To circumvent this, 
after each iteration \fun{UnsetDeformation()} is called. To not overshoot anything, 
instead of adding the entire \var{gfX} to \var{gfset}, it is scaled by a number divided 
by its norm. That way one can make sure, that each iteration deforms the mesh in small, 
similar sized steps.  The weights $\alpha$ and $\beta$, and $\gamma$ for the volume and 
barycenter constraints are recalculated or conditionally updated through the iterations. \\

\begin{algorithm}
	\setstretch{1.5}
	\renewcommand{\thealgorithm}{}
	\caption{PDE Constrained Shape Optimization in NGSolve}
	\begin{algorithmic}[1]
	\State \fun{resetDeformation}
	\State \fun{initializeParameters} $\alpha$, $\beta$, $\gamma$ \Comment{Aug. Lag. weights for
	$\mathrm{vol}(\Omega_{\mathrm{i}})$, $\mathrm{bc}_x(\Omega_{\mathrm{i}})$, $\mathrm{bc}_y(\Omega_{\mathrm{i}})$}
	\For{$\mathrm{i} < \mathrm{iter}_{\mathrm{max}}$}
		\State SolveStokes() \Comment{Solve Stokes on $\Omega_i$}
		\State SolveDeformationEquation() \Comment{Solve Auxiliary Problem on $\Omega_i$, yields $X$}
		\State Evaluate \fun{gfxbndnorm} = $ || X ||_{\mathrm{L}^2(\Gamma_{\infty , \mathrm{i}})} $
		\State Evaluate \fun{ScalingParamter} = $ \frac{0.01}{|| X ||_{\mathrm{L}^2(\Omega_{\mathrm{i}})}} $
		\If{\fun{gfxbndnorm} $<$ $\varepsilon$} 
			\State Increase $\alpha$, $\beta$, $\gamma$
		\If{\fun{parametersTooBig}}
			\State \fun{break}
		\EndIf
		\EndIf
		\State Set $\Omega_{i+1}$ =  $\Omega_i - X \cdot$ \fun{ScalingParameter} \Comment{Gradient Descent Step}
	\EndFor
	\end{algorithmic}
\end{algorithm}