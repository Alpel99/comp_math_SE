\subsection{Iteration}
In this chapter we want do describe the process of one iterations which gets repeated time after time until the optimal shape is achieved.

There are a few parameters that are recalculated or conditionally updated through the iterations. This includes the current volume and barycenter values, their respective scalings $\alpha$ and $\beta$, and $gamma$ for the Cauchy Riemann terms that ensure a better mesh quality.

Before we start with the iteration process we reset all possible variables:
This includes the \var{gfset}, resetting the scene, and to reinitialize the parameters for the geometric constraints. If this step is skipped, one could start with weird variable values that lead to drastically different results.

\begin{lstlisting}[language=Python, title=Reset before iteration, label=lst:reset]
gfset.Set((0,0))
mesh.SetDeformation(gfset)
scene.Redraw()
updateParams()
alpha0 = 1e-4
beta0 = 1e-0
gamma0 = 1e2
alpha.Set(alpha0)
beta.Set(beta0)
gamma.Set(gamma0)
iter_max = 750
\end{lstlisting}

This iteration is bounded by a maximum number of steps, even though it is also be possible to take some measure on the \var{gfX} and determine a stop by it.
This still requires parameter tuning, especially because some measures could yield different results depending on the mesh width.
Other possibilities for breaks in the loop are a maximum number for the scaling parameters e.g. $\alpha$ or a very small difference in the drag from one iteration to the next.

Each iteration then starts by calling \fun{SetDeformation(gfset)}, and redrawing needed scenes. This would also be the place to gather data, like the current drag or area of the mesh, etc.

Afterwards we start calculating towards the next step:
We Assemble the state equation bilinear form and solve it over this newly deformed mesh. Following this we assemble the linear and bilinear form for the shape derivative and solve for a new deformation. After this is done, one can already use \fun{UnsetDeformation()} and the last thing to do is updating values.

We scale the \var{gfX} to its desired magnitude and subtract it from \var{gfset}. At this point it is also wise to check for some measure of close we are to a good solution and for example increase the parameters for the geometric constraints.

The entire code inside the loop looks like this:

\begin{lstlisting}[language=Python, title=Iteration, label=lst:loop]
		mesh.SetDeformation(gfset)
		scene.Redraw()
		data.append(vol.Get())
		a.Assemble()
		solveStokes()
		b.Assemble()
		dJOmega.Assemble()
		SolveDeformationEquation()
		updateParams()
		mesh.UnsetDeformation()
		gfxnorm = Norm(gfX.vec)
		scale = 0.01 / gfxnorm
		
		if(gfxnorm < 1e-5):
			if alpha.Get() < 1:
				increaseParams(2,True)
			else:
				break
		gfset.vec.data -= scale * gfX.vec
\end{lstlisting}

It is also possible to implement some sort of line search for the step size, similar to the armijo rule. This way one can do less iterations but take better/bigger steps, being more efficient in regards to computational efforts. More to this can be read in ...