\section{Augmented Lagrangian and Geometric Constraints}

In the previous chapter, the differentiability of the shape function with respect to the perturbation 
parameter $t$ was recapitulated \cite{nearly_conformal_paper}. One can now employ a minimization problem
which yields an approximative solution. This is where the Augmented Lagrangian is introduced. Since differentiability was shown,
one can show that the negative derrivative of the Augmented Lagrangian is an approximate minimizer to the 
Lagrangian (\ref{parametrized_lagrangian}), for elaborations on Augmented Lagrangians see Sturm Lecture Notes \cite{lecture_notes_sturm}. 
One considers again the Lagrangian (\ref{parametrized_lagrangian}) in the unparametrized way which can be evaluated on the FEM mesh:
\begin{equation}\label{basic_lagrangian}
	%L(\Omega,u,p) = 
    \mathcal{L}(\Omega) = \frac{1}{2} \int_{\Omega} \mathrm{D}\mathbf{u}: \mathrm{D} \mathbf{u} \, \mathrm{dx} + 
	\int_{\Omega} \nabla \mathbf{u} : \nabla \mathbf{u} \, \mathrm{dx} + \int_{\Omega} \mathrm{div}(\mathbf{u})p \, \mathrm{dx} +
	 \int_{\Omega} \mathrm{div}(\mathbf{u})p \, \mathrm{dx}
\end{equation}
If one minimizes this problem, one can observe: the numerical scheme will just make the obstacle smaller since this will 
result in a minimization of the drag as well. This is not a trivial result but also not a result of interest. Therefore one analyzes the 
volume and barycenter of the obstacle and formulates terms that penalize deviations from the initial values of the volume and barycenter.
\begin{align}
    \mathrm{vol}(\Omega) &= \int_{\Omega} 1 \ \mathrm{dx} \in \mathbb{R} \label{vol_equation}\\
    \mathcal{L}^2_{\mathrm{vol}}(\Omega) &= \alpha \ \Big( \mathrm{vol}( \mathbf{T}_{t}(\Omega) )-\mathrm{vol}(\Omega_0) \Big) ^2
\end{align}
The quantity $\mathcal{L}_{\mathrm{vol}}(\Omega)$ is a quadratically penalized term that is easy to derrive and is used to formulate 
the Augmented Lagrangian. If the deformed domain $\Omega$ yields an obstacle of identical volume, the term is zero, deviations 
from the initial obstacle volume are penalized. The same is done for the barycenter of the obstacle. 
\begin{align}
    \mathrm{bc}(\Omega) =
    \frac{1}{\mathrm{vol}(\Omega)}\int_{\Omega} \mathbf{x} \ \mathrm{d} \mathbf{x} \in \mathbb{R}^2
\end{align}
Since it is a vectorial quantity, the integral is decomposed into its components such that it yields a scalar quantity:
\begin{align}
    \mathrm{bc}_x(\Omega) &=
    \frac{1}{\mathrm{vol}(\Omega)}\int_{\Omega} x \ \mathrm{dx} \, , \quad 
    \mathrm{bc}_y(\Omega) =
    \frac{1}{\mathrm{vol}(\Omega)}\int_{\Omega} y \ \mathrm{dx} \label{bc_equation}\\
	\mathcal{L}^2_{\mathrm{bc}}(\Omega) &= \beta \Big( \mathrm{bc}_x( \mathbf{T}_{t}(\Omega) )-\mathrm{bc}_x(\Omega_0) \Big)^2 + 
	\gamma\Big( \mathrm{bc}_y( \mathbf{T}_{t}(\Omega) )-\mathrm{bc}_y(\Omega_0) \Big)^2
\end{align}
Finally, the quadratically penalized Augmented Lagrangian:
\begin{align}\label{final_aug_lagrange}
	\mathcal{L}_{\mathrm{aug}}(\Omega) =  \mathcal{L}(\Omega) + \mathcal{L}^2_{\mathrm{vol}}(\Omega) + \mathcal{L}^2_{\mathrm{bc}}(\Omega)
\end{align}
The parameters $\alpha$, $\beta$ and $\gamma$ are used to weigh the quadratically penalized terms and vary them dynamically while
iterating. \\

Remark: The quadratically penalized terms do not evaluate the surface and barycenter of the obstacle and its deviations, but rather
of the entire domain $\Omega$, but since the obstacle is in the center and the entire geometry symmetric, the obtained quantities are representative of
the surface and barycenter of the obstacle as well.



