\subsection{The Lagrangian, State Equation and Geometric Constraints}
%To solve this the method of Lagrange multipliers is used to find minima or maxima of a function

It is shown in Sturm (2015a) that the shape derivative for a nonlinear PDEconstrained shape optimisation problem can be computed as the derivative of the Lagrangian with respect to the perturbation parameter.\cite{fully_semi_paper_sturm}

This Lagrangian is given by the stokes equation with right hand side zero and the shape function $J$.
For the Stokes equation there are only u and p used, not v and q.

\begin{equation}\label{eq:lagrangian}
    L(\Omega,u,p) = \int_{\Omega} \nabla u : \nabla u \, dx + \int_{\Omega} \mathrm{div}(u)p \, dx + \int_{\Omega} \mathrm{div}(u)p \, dx + \int_{\Omega} Du : Du \, dx
\end{equation}

Additional to these, there geometrical constraints should be  enforced: On the one hand the volume has to be constant, because otherwise the obstacle just shrinks to nothing. This would obviously minimize the dissipated energy, but not tell anything useful for the solution.

Secondly, the barycenter of the obstacle should stay constant, because otherwise it could just try to move out of the mesh to minimize the shape function.

Since in our approach only the mesh for the flow outside of the obstacle is generated, we assign those geometric constraints on this mesh and not on the obstacle itself. In general that does not change anything about the problems with these constraints in itself. This is because if the barycenter and volume of the surrounding mesh stay constant, the same is true for the obstacle.