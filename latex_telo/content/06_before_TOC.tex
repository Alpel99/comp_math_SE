\pagenumbering{roman}

\section*{Abstract}
In the underlying work, a complete shape optimization procedure is done with The python Finite Element Method 
library NGSolve. As an initial problem, the linear stationary Stokes flow around a cylinder in a rectangular
domain is considered. Since the velocity vectorfield is available as a solution of the weak stokes problem, 
the energy dissipation $J(\Omega)$ can be evaluated over the entire domain as an integral. The goal of the 
optimization problem is to minimize said energy dissipation, by perturbing the domain. The formulation of the
functional $J(\Omega)$ and proof of existence of its derivative with respect to the domain perturbation 
$\mathrm{d}J(\Omega)(X)$ is sufficient, to formulate an augmented Lagrangian. The derivative of the Augmented
Lagrangian can be used to minimize the Augmented Lagrangian, which is an approximative solution to the 
initial minimization of $J(\Omega)$. Additionally, since the minimization is done iteratively, one solves 
another PDE problem on the same domain which is the auxiliary problem. Its result is the perturbation
of the domain in minimizing direction of $J(\Omega)$. The auxiliary problem is posed in such a way, that 
each perturbation is near conformal, or of near angle preserving character. This near conformality results
in a mesh that is still of acceptable quality after hunderds of iterations. The succesful implementation in 
NGSolve with acceptable convergence behaviour and good agreement with literature are documented in this 
seminary paper. The agreement with the literature being, that the shape of the cylinder is actually 
reshaped to an ogive with 45 degree tips and the near conformality of the mesh over all iterations is 
guaranteed.
