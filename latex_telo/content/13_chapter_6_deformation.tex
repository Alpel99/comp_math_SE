\subsection{Auxiliary Problem}
The descent direction in the gradient descent method, is the direction $-X$.
In the previous chapter, a formulation for $\mathrm{d}\mathcal{L}_{\mathrm{aug}}(\Omega)(X)$
was derived. As a next step one needs to formulate an auxiliary problem: since it is demanded,
that the functional $\mathcal{L}_{\mathrm{aug}}(\Omega)$ is minimized, one follows the 
negative gradient $-X$. This yields a PDE which can be solved in a weak sense with FEM in NGSolve.
The weak formulation of the auxiliary problem for $X$ reads as follows \\

find $X \in [H(\Omega)]^2$ such that
\begin{align}
	\mathrm{d}\mathcal{L}_{\mathrm{aug}}(\Omega)(X) = -(X,\varphi)_H \quad \forall \varphi \in H(\Omega).
\end{align}

Sturm et. al. \cite{nearly_conformal_paper} have proposed different spaces $H$ for the auxiliary problem
and investigated their impact analytically. The main criterion being, that the bi-linear 
form $(X,\varphi)_H$ is positive definite,
to guarantee that the direction $-X$ is indeed negative. For the case $H=H^1(\Omega)$
\begin{align}
	(X,\varphi)_{H^1(\Omega)} = \int_{\Omega} \, X \, \varphi \, + \mathrm{D}X:\mathrm{D}\varphi \, \mathrm{dx}.
\end{align}

\subsection{The Vectorfield $X$ as a Conformal Mapping}
For the notion of conformality, one introduces the Cauchy-Riemann Equations. This is used here,
to omit remeshing of the domain or severe degradation of the elements.
This because remeshing is relatively expensive computation wise and element degradation
would lead to a large local error of the Stokes flow solution.
If the vectorfield $X$ satisfies the Cauchy-Riemann Equations it is a holomorphic injective 
transformation, which is conformal. Where $X = (X_1, X_2) \in [C^1(\Omega)]^2$

\begin{equation}\label{CR_eq}
	\begin{aligned}
		\partial_x X_1 &= \partial_y X_2 \\
		\partial_y X_1 &= -\partial_x X_2. \\
	\end{aligned}
\end{equation}
The linear operator $\mathcal{B}$ is used for more compact notation of the 
Cauchy-Riemann equations (\ref{CR_eq})
\begin{equation}\label{CR_B}
	\begin{aligned}
		\mathcal{B} = 
		\begin{pmatrix}
			-\partial_x & \partial_y\\
			\partial_y & \partial_x
		\end{pmatrix} ,\quad [C^1(\Omega)]^2 \rightarrow [C^0(\Omega)]^2.
	\end{aligned}
\end{equation}
With introduction of $\mathcal{B}$, one can write the CR-Equations (\ref{CR_eq}) as
\begin{align*}
	\mathcal{B}X=0.
\end{align*}
Now one can use the CR-Equations to adjust the auxiliary problem to yield nearly conformal 
mappings. Where $|| \, . \, ||_P: \, P \rightarrow \mathbb{R}$ is a
norm on a Hilbert space $P$ and $\mathcal{B}:H \rightarrow P$ is a linear continuous operator 
in $P$  such that $\mathcal{B}(H) \subset P$, find $X \in [H(\Omega)]^2$
\begin{align}\label{eq_conf_aux}
	\mathrm{d}\mathcal{L}_{\mathrm{aug}}(\Omega)(X) = \alpha (\mathcal{B}X,\mathcal{B}\varphi)_P
	+ (X,\varphi)_H, \quad \alpha \in \mathbb{R}, \quad \forall \varphi \in [H(\Omega)]^2.
\end{align}
The parameter $\alpha$ can now be used to weigh the conformality of the mapping. For higher 
$\alpha$ the mapping is more conformal.


