\subsection{Auxiliary Problem: Deformation Field $X$}
The deformation, as described in the mathematical sense is the $X$ in $\Omega_t := (Id + tX )(\Omega)$.
Its according space is \obj{H1(mesh, order=2, dim=2)}, with dirichlet boundaries on the outsides of the square.

This variable in our ngsolve implementation is called \var{gfX}. This is a two-dimensional \obj{GridFunction} which is computed from the shape derivative and stores the information how to deform the mesh at each node.

The shape derivative is the linear form \var{dJOmega} in our code. Since the goal is, to make steps in a descent direction (negative), we have to make sure we only calculate according solutions.
This is done by setting the bilinear form of this problem to the $H^1$ norm.

\begin{equation}
	\partial J(\Omega) = \int(\phi\cdot X)+(\nabla\phi\cdot \nabla X) \, dx
\end{equation}

This always yields a positive value, if we subtract instead of add this value, we minimize the shape function.

\begin{comment}
% is this even useful?
\begin{lstlisting}[language=Python, title=Solving For The Deformation, label=lst:deformation_solve]
	def SolveDeformationEquation():
	rhs = gfX.vec.CreateVector()
	rhs.data = dJOmega.vec - b.mat * gfX.vec
	update = gfX.vec.CreateVector()
	update.data = b.mat.Inverse(VEC.FreeDofs()) * rhs
	gfX.vec.data += update
\end{lstlisting}
\end{comment}

Since the minimization is done in iterations, we have to keep track of the previous deformations. This is done in ngsolve by adding the parts of \var{gfX} to another variable called \var{gfset}. This \var{gfset} is then always used to call \fun{SetDeformation()} on the mesh.
With each call, this adds the \var{gfset} onto the mesh. To circumvent this, after each iteration \fun{UnsetDeformation()} is called.

To not overshoot anything, instead of adding the entire \var{gfX} to \var{gfset}, it is scaled by a number divided by its norm. That way we can make sure, that each iteration deforms the mesh in small, similar sized steps.
There is still one inconsistency: the \var{gfX} can also deform nodes inside the mesh, which change nothing for the real solution, but count towards this norm. That problem is solvable, by integrating this \obj{GridFunction} over its boundary. Since the outside of our square are dirichlet boundaries, this way only the changes on the obstacle are measured. Another important thing to be aware of is, that symmetric deformations around an obstacle might cancel each other out in the integral. This is circumvented by calculating with the squared values.




